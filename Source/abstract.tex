\myabstract{%
	% Zusammenfassung
	\change{Transate abstract}
	\lipsum[1]
	An academic abstract typically outlines four elements relevant to the completed work:
	\begin{itemize}[noitemsep]
		\item The research focus (i.e. statement of the problem(s)/research issue(s) addressed)
		\item The research methods used (experimental research, case studies, questionnaires, etc.)
		\item The results/findings of the research
		\item The main conclusions and recommendations
	\end{itemize}
	}{%
	% Abstract
	Wearable devices such as smartwatches or activity trackers with embedded sensors are capable of exchanging data with other connected devices.
	This data will often be transferred to the manufacturer or processed directly on a connected smartphone in order to provide user feedback based on the analyzed data.
	Almost every wearable device offers third-party developers a way to gain (at least partial) access to the gathered sensor data, allowing custom applications to process them. 

	In the course of this work, the availability of APIs\footnote{Application Program Interfaces, provided by the manufacturer} on different wearables and how likely they can be used to transfer and process sensor data in real-time has been evaluated, as well as the suitability of current devices running proprietary operating systems created by Apple, Google, Jawbone and Microsoft.
	In addition, an in-depth look at possible implementations for real-time processing of sensor data from devices running Android Wear is included.

	\improvement{Add results}
	}