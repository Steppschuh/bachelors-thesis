\myabstract{
	Tragbare Geräte, wie Smartwatches oder Fitness Tracker mit integrierten Sensoren, sind in der Lage, Daten mit anderen verbundenen Geräten auszutauschen.
	Diese Daten werden häufig an die Hersteller übertragen oder direkt auf einem verbundenen Smartphone verarbeitet.
	So kann Nutzern Feedback basierend auf den gemessenen Daten gegeben werden.
	Nahezu jedes dieser tragbaren Geräte bietet Entwicklern die Möglichkeit, auf die Messwerte zuzugreifen, um diese mit eigener Software zu verarbeiten.

	Im Rahmen dieser Arbeit wurde die Verfügbarkeit von APIs\footnote{Application Program Interfaces, Schnittstellen zur Anwendungsprogrammierung} auf verschiedenen Geräten und dessen Eignung zur Übertragung von Sensor Daten in Echtzeit evaluiert.
	Funktionale Implementierungen für die Datenübertragung zwischen tragbaren Geräten und Smartphones basierend auf der Android Plattform wurden präsentiert.
	Die Leistung, Effizienz und der Akkuverbrauch wurden basierend auf Messungen analysiert.
}{
	Wearable devices such as smartwatches or activity trackers with embedded sensors are capable of exchanging data with other connected devices.
	This data will often be transferred to the manufacturer or processed directly on a connected smartphone in order to provide user feedback based on the analyzed data.
	Almost every wearable device offers third-party developers a way to gain (at least partial) access to the gathered sensor data, allowing custom applications to process them. 

	In the course of this work, the availability of APIs\footnote{Application Program Interfaces, provided by the manufacturer} on different wearables and how likely they can be used to transfer and process sensor data in real-time has been evaluated.
	We have presented functional implementations for transferring data between wearables and mobile devices powered by the Android platform. Performance, efficiency and battery impact have been analyzed using benchmarks.
}