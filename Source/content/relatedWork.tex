\section{Related Work}
\label{sec:relatedwork}

\subsection{Project Abacus}
\begin{itemize}[noitemsep]
	\item What's different
\end{itemize}

Wang et al. evaluated whether it is possible to figure out what a user is typing on a keyboard, just by looking at the sensor data that smartwatches produce.
In their papger ``MoLe: Motion Leaks through Smartwatch Sensors''\cite{paper:motionleaks}, they described how they analysed peoples typing behaviour in order to identify possible information leaks when users type while wearing a smartwatch.
They also developed a system for the Samsung Gear Live smartwatch (one of the first devices running Android Wear) that resembles motion data to commonly used english words.
Although MoLe turned out to be able to detect typed words with an high accuracy, they used the wearable only to record the sensor data.
Files containing the data were exported from the watch and processed on more powerful devices and not on mobile devices, which does not meat our project's requirements.

In ``Real-Time Sensing on Android''\cite{paper:realtimesensing}, Yin Yan et al. examined Android’s sensor architecture and whether it is suitable for use in a real-time context.
They took an in-depth look at the very low-level implementations of the Android |SensorManager|\cite{androiddocs:sensormanager}, including the kernel, HAL\footnote{Hardware Abstraction Layer}, and the |SensorService| (which polls raw sensor data through the HAL).
Their research showed that Android's sensor architecture does not provide predictable sensing.
It does not have any priority support in sensor data delivery, because all sensor data follows a single path from the kernel to apps.
Also, the amount of time it takes to deliver sensor data is unpredictable because Android relies heavily on polling and buffering.
For our project however, we worked around these limitations by implementing algorithms that abstracted from the data frequency and delay.


\clearpage