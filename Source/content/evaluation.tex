\section{Evaluation}
\label{sec:evaluation}

\subsection{Benchmarks}
\label{sec:evaluation:benchmarks}
In order to measure how performant different implementations are, we created different benchmarks.
These helped us to evaluate which parts of our solution require optimization.

\subsubsection{Setup}
asdf


\subsubsection{Data Transmission Delay}
When transmitting data, we had to


In order to reduce the serialization overhead, we collect sensor data in |DataBatches|\cite{sensordatalogger:databatch} before transferring it to a mobile device.
This approach drastically improves the \textit{delay per sent byte} ratio, because we have to serialize and deserialize less messages and less meta data.
However, it also delays the data depending on the batch capacity.
Just like buffering a stream, this is a trade-off between beeing less efficient or beeing less real-time.

For the measurements in table \ref{table:benchmark:transmissiondelay:50ms}, we batched sensor data for \textbf{50 milliseconds} before transferring it:

\begin{table}[H]
	\begin{tabular}{rrrl}
		delay in ns       & bytes             & delay / bytes     & comment \\ \hline

		1,266,250,000     & 200               & \textasciitilde6,331,250        & \textasciitilde1 update (|SENSOR_DELAY_NORMAL|) \\
		1,271,160,000     & 482               & \textasciitilde2,637,261        & \textasciitilde2 updates (|SENSOR_DELAY_UI|) \\
		1,285,510,000     & 1,056             & \textasciitilde1,217,339        & \textasciitilde6 updates (|SENSOR_DELAY_GAME|) \\
		1,306,400,000     & 3,599             & \textasciitilde362,989          & \textasciitilde24 updates (|SENSOR_DELAY_FASTEST|) \\
	\end{tabular}
	\caption{Transmission delay, 50ms batches}
	\label{table:benchmark:transmissiondelay:50ms}
\end{table}

When using |SENSOR_DELAY_NORMAL|, the sensor reports only about one update during the batching duration.
This basically results in no improvement at all because we still have to deal with the serialization overhead for each update.
By collecting more updates in the same time frame (in this case by switching to |SENSOR_DELAY_FASTEST|), we were able to boost the efficency by 94.3\% while only raising the delay by 3.2\%.

We wanted to improve even further and increased the batching duration to 500 milliseconds, as measured in table \ref{table:benchmark:transmissiondelay:500ms}:

\begin{table}[H]
	\begin{tabular}{rrrl}
		delay in ns       & bytes             & delay / bytes     & comment \\ \hline

		1,329,120,000     & 625               & \textasciitilde2,126,592        & \textasciitilde3 updates (|SENSOR_DELAY_NORMAL|) \\
		1,331,970,000     & 1,340             & \textasciitilde994,007          & \textasciitilde8 updates (|SENSOR_DELAY_UI|) \\
		1,373,370,000     & 8,262             & \textasciitilde166,227          & \textasciitilde28 updates (|SENSOR_DELAY_GAME|) \\
		1,412,810,000     & 16,951            & \textasciitilde83,346           & \textasciitilde118 updates (|SENSOR_DELAY_FASTEST|) \\
	\end{tabular}
	\caption{Transmission delay, 500ms batches}
	\label{table:benchmark:transmissiondelay:500ms}
\end{table}

Increasing the duration resulted in more updates per transferred |DataRequestResponse|\cite{sensordatalogger:datarequestresponse}.
Compared to the first measurement in table \ref{table:benchmark:transmissiondelay:50ms}, we were able to transfer 84 times more bytes at the cost of only 147 milliseconds.

For some use cases, less frequent data might be sufficient. Instead of altering the reporting delay of the sensor, we can also sent data from multiple sensors simultaneously. Table \ref{table:benchmark:transmissiondelay:multiple} shows how multiple, 3-dimensional sensors perform:

\begin{table}[H]
	\begin{tabular}{rrrl}
		delay in ns       & bytes             & delay / bytes     & comment \\ \hline

		1,452,400,000     & 16,690            & \textasciitilde87,022           & \textasciitilde116 updates,  1 sensor \\
		1,489,100,000     & 22,819            & \textasciitilde65,257           & \textasciitilde246 updates,  2 sensors \\
		1,752,420,000     & 60,679            & \textasciitilde28,880           & \textasciitilde512 updates,  4 sensors \\
		2,842,640,000     & 177,495           & \textasciitilde16,015           & \textasciitilde1504 updates, 8 sensors \\
	\end{tabular}
	\caption{Transmission delay, multiple sensors}
	\label{table:benchmark:transmissiondelay:multiple}
\end{table}

Keeping in mind that this transfer is performed multiple hundreds or thousands times, these efficency improvements sum up and save a lot of computing and battery power.

\begin{itemize}[noitemsep]
	\item Transferred data per second
	\item Delay
	\item Serialization overhead
	\item Battery impact
	\item Different devices / API versions
\end{itemize}

\clearpage